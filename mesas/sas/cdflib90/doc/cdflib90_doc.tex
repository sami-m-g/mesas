%\documentclass[12pt,pdftex]{article}
\documentclass[12pt,dvips]{article}
\usepackage{color}
\addtolength{\oddsidemargin}{-0.5in}
\addtolength{\textheight}{1.0in}
\addtolength{\textwidth}{0.5in}
\setlength{\parindent}{0in}
\newcommand{\meta}[1]{{\ensuremath{<}}#1{\ensuremath{{>}}}}
\newcommand{\us}{\underline{ }}

\newcommand{\range}[2]{\hfill Range: \ensuremath{\left[ #1:#2
\right]}\\}
\newcommand{\inrange}[2]{\hfill Input Range: \ensuremath{\left[ #1:#2
\right]}\\}
\newcommand{\searchrange}[2]{\hfill Search Range: \ensuremath{\left[ #1:#2
\right]}\\}

% numbers

\newcommand{\sdf}{10^{-3}}
\newcommand{\bdf}{10^{10}}
\newcommand{\sprob}{10^{-10}}
\newcommand{\bprob}{1-\sprob}
\newcommand{\immense}{10^{100}}
\newcommand{\rimmense}{10^{-100}}
\newcommand{\pfour}{10^{4}}
\newcommand{\spnonc}{0}
\newcommand{\bpnonc}{10^4}
\newcommand{\bbig}{10^10}

\renewcommand{\labelitemi}{\ensuremath{\bullet}}
\renewcommand{\labelitemii}{\ensuremath{\circ}}
\renewcommand{\labelitemiii}{\ensuremath{\odot}}
\definecolor {blue} {rgb} {0, 0, 1}
\definecolor {red} {rgb}  {1, 0, 0}
\definecolor {green} {rgb} {0, 1, 0}
\definecolor{Orange}        {cmyk}{0,0.61,0.87,0}

 \definecolor{Orange}{cmyk}{0,0.61,0.87,0}
 \newcommand{\mysection}[1]{\color{blue}
             \section{#1} \normalcolor}
 \newcommand{\mysubsection}[1] {\color{green}
             \subsection{#1} \normalcolor}
 \newcommand{\mysubsubsection}[1]{\color{Orange}
             \subsubsection{#1} \normalcolor}
\newcommand{\myitem}[1]{\item{\bf \color{Violet} #1 \normalcolor}}
\definecolor{Violet}        {cmyk}{0.79,0.88,0,0}

% hyperref stuff
% \usepackage[bookmarks,
%              bookmarksopen,
%              pdftex=true]{hyperref}


\begin{document}

% Title Page

\vspace*{\fill}

{\Large \centering \textcolor {red} {CDFLIB90\\
\textcolor {blue}{Fortran 95 Routines for Cumulative Distribution Functions,}\\
\textcolor {Orange}{Their Inverses, and More.}\\
\textcolor {green} {User's Guide}\\}}

\vspace{0.5in}

{\large \centering \textcolor{Orange}{Barry W. Brown\\
James Lovato\\Kathy Russell\\}}

\vspace{0.5in}

\begin{center}
{\large \textcolor{Orange}{Conversion to Fortran 95\\
Dan Serachitopol\\}}
\end{center}

\vspace{0.5in}

This work supported by grant CA11672 from the National Cancer
Institute.  Copyright 2002 to:\\

\begin{quote}
      The University of Texas M. D. Anderson Cancer Center\\
      Department of Biomathematics, Box 237\\
      1515 Holcombe Boulevard\\
      Houston, TX 77030\\
\end{quote}

\vspace{0.5in}

Contact: Barry W. Brown, bwb@mdanderson.org

\vspace*{\fill}

\pagebreak

\mysection{Technicalities}

\mysubsection{Obtaining the Code}

The source for this code (and all code written by this group) can be
obtained from the following URL:\\

{\bf http://odin.mdacc.tmc.edu/anonftp/\\}

\mysubsection{Legalities}

We place  our efforts  in writing this  package in the  public domain.
However,  code from  ACM publications  is  subject to  the ACM  policy
(below).

\mysubsection{References}

\mysubsubsection{Incomplete Beta}

DiDinato, A. R. and Morris,  A.   H. (1993)  ``Algorithm 708: Significant
Digit Computation of the Incomplete  Beta  Function Ratios.''  {\em ACM
Trans. Math.  Softw. 18}, 360-373.

\mysubsubsection{Incomplete Gamma}

DiDinato, A. R. and Morris, A. H. (1986) ``Computation of the  incomplete
gamma function  ratios  and their  inverse.''   {\em ACM  Trans.  Math.
Softw. 12}, 377-393.

\mysubsubsection{Cumulative Normal}

Cody, W.D. (1993). ``ALGORITHM 715: SPECFUN - A Portable FORTRAN
Package of Special Function Routines and Test Drivers''
{\em ACM Trans. Math. Softw. 19}, 22-32.

\mysubsubsection{Inverse Normal}

``Algorithm AS241'' (1988) {\em Appl. Statist. 37}, NO. 3, 477-484.

\mysubsubsection{Finding a Zero of a Monotone Function}

Alefeld,  G.  E.,  Potra,  F.  A., Shi,  Y.  (1995)  ``Algorithm  748:
Enclosing  Zeros  of  Continuous   Functions.'',  by  G.  E.  Alefeld,
F.  A. Potra, YiXun  Shi, {\em ACM  Trans.  Math. Softw.,
21}, No. 3, 327-344

\mysubsection{ACM Policy on Use of Code}

Here  is the  software Policy of  the  ACM.

\begin{quote}

     Submittal of  an  algorithm    for publication  in   one of   the  ACM
     Transactions implies that unrestricted use  of the algorithm within  a
     computer is permissible.   General permission  to copy and  distribute
     the algorithm without fee is granted provided that the copies  are not
     made  or   distributed for  direct   commercial  advantage.    The ACM
     copyright notice and the title of the publication and its date appear,
     and  notice is given that copying  is by permission of the Association
     for Computing Machinery.  To copy otherwise, or to republish, requires
     a fee and/or specific permission.

     Krogh, F.  (1997) ``Algorithms  Policy.''  {\em ACM  Tran.  Math.
     Softw.  13}, 183-186.

\end{quote}

We do not know the policy  of the Royal Statistical Society; they have
discontinued publishing  algorithms.  However,  they made a  number of
these  programs available  on Statlib  on condition  that there  be no
charge for their distribution.

Here is our standard disclaimer.

\begin{quote}

{\centering NO WARRANTY\\}

WE PROVIDE ABSOLUTELY  NO WARRANTY  OF ANY  KIND  EITHER  EXPRESSED OR
IMPLIED,  INCLUDING BUT   NOT LIMITED TO,  THE  IMPLIED  WARRANTIES OF
MERCHANTABILITY AND FITNESS FOR A PARTICULAR PURPOSE.  THE ENTIRE RISK
AS TO THE QUALITY AND PERFORMANCE OF THE PROGRAM IS  WITH YOU.  SHOULD
THIS PROGRAM PROVE  DEFECTIVE, YOU ASSUME  THE COST  OF  ALL NECESSARY
SERVICING, REPAIR OR CORRECTION.

IN NO  EVENT  SHALL THE UNIVERSITY  OF TEXAS OR  ANY  OF ITS COMPONENT
INSTITUTIONS INCLUDING M. D.   ANDERSON HOSPITAL BE LIABLE  TO YOU FOR
DAMAGES, INCLUDING ANY  LOST PROFITS, LOST MONIES,   OR OTHER SPECIAL,
INCIDENTAL   OR  CONSEQUENTIAL DAMAGES   ARISING   OUT  OF  THE USE OR
INABILITY TO USE (INCLUDING BUT NOT LIMITED TO LOSS OF DATA OR DATA OR
ITS ANALYSIS BEING  RENDERED INACCURATE OR  LOSSES SUSTAINED  BY THIRD
PARTIES) THE PROGRAM.

(Above NO WARRANTY modified from the GNU NO WARRANTY statement.)
\end{quote}

\mysection{Introduction to CDFLIB90}

\mysubsection{Modules}

CDFLIB90 contains routines that calculate the cumulative distribution
function (CDF), the inverse CDF, and the values of one parameter of
the distribution given the value of the CDF and of the other parameters.

{\bf REAL TYPE.}   The only type of Fortran REAL  used in this library
is DOUBLE PRECISION, which we term {\bf dpkind}.

CDFLIB90 consists of Fortran 95 modules, named as follows:

{\centering \bf cdf\_\meta{distribution}\_mod\\}

Each program unit using one or more of these module must have the
appropriate USE statement(s).

The following  table shows the  possible values of  {\bf \meta{distribution}}
and the corresponding statistical distribution.
\hspace{0.5in}

\begin{center}
\begin{tabular}{ll}
\hline
{\bf \meta{distribution}} & {\bf Statistical} \\
& {\bf Distribution}\\ \hline
beta & beta \\ \hline
binomial & binomial \\ \hline
chisq & chi-squared \\ \hline
f & f \\ \hline
gamma & gamma \\ \hline
nc\_chisq & noncentral \\
& chi-squared \\ \hline
nc\_f & noncentral f \\
\hline
nc\_t & noncentral t \\
\hline
neg\_binomial & negative \\
& binomial \\ \hline
normal & normal \\ \hline
poisson & poisson \\ \hline
t & t \\ \hline

\end{tabular}
\end{center}

\mysubsection{Routines}

Each module contains four user accessible (PUBLIC) routines:

\begin{description}

\item SUBROUTINE CDF\_\meta{distribution}( WHICH, CUM, CCUM, X,
\meta{PARAMS}, STATUS, CHECK\_INPUT.)
Calculates the value  of any one of its parameters  from the values of
the others.

\item    REAL   (dpkind)   FUNCTION    CUM\_\meta{distribution}(X,
\meta{PARAMS}, STATUS, CHECK\_INPUT.)  Calculates argument CUM.

\item    REAL    (dpkind) FUNCTION    CCUM\_\meta{distribution}(X,
\meta{PARAMS}, STATUS, CHECK\_INPUT) Calculates argument CCUM.

\item  REAL (dpkind) FUNCTION  INV\_\meta{distribution}(  CUM, CCUM,
\meta{PARAMS}, STATUS, CHECK\_INPUT)

\end{description}

\mysubsection{Arguments}

\begin{description}

\myitem{INTEGER, INTENT(IN):: WHICH.} Determines which parameter is to be
calculated from the values of the other.

\myitem{REAL (dpkind) :: CUM, CCUM.} The CDF and complement of the CDF,
1-CUM.

\myitem{REAL (dpkind) :: X.} The name of this argument varies with
the distribution.

The upper  limit of integration or  summation for the  CDF.  The lower
limit is the lower limit of the support of the distribution.

Note that X  is always DOUBLE PRECISION, even when  it appears that it
should  take only  integer values,  as in  the  binomial distribution.
This  is  because  discrete  distribution is  embedded  in  continuous
distributions.  The binomial and negative binomial are calculated as a
special  case of the  beta distribution;  the Poisson  distribution is
calculated as a special case of the gamma distribution.

One consequence: when X is computed,  it need not and usually does not
take on an integer value for discrete distributions.

\myitem{REAL  (dpkind)   ::  \meta{PARAMS}.}   The   parameters  of  the
distribution, e.g.,  the mean and variance of  the normal distribution
or the degrees of freedom of the t distribution.

\myitem{INTEGER, INTENT(OUT), OPTIONAL ::  STATUS.}  The status of the
calculation.  {\bf Although this  argument is optional, it is strongly
  recommended that  all code include it  and its value  checked by the
  calling program.}

Zero indicates  success.  

A small negative number indicates that the argument whose order is the
absolute value of status is out of bounds.

A small  positive value generally indicates that  input arguments that
should add (approximately) to one don't.

The value `10' indicates that the cumulative function is obtained from
another  distribution   (e.g.,  the  binomial  is   calculated  as  an
incomplete beta)  and there  was an error  in the other  distribution. 
This really should not occur.

Should the status  be any of the values above  and the argument STATUS
is not present, then the program aborts (STOP) with an error message.

A value of  50 indicates that the answer, should  it exist exceeds the
upper search bound  of the routine; a value of  -50 indicates that the
lower search  bound is exceeded.  In  either case, the  upper or lower
search bound respectively is returned as the answer.  The program does
not abort  in this case  should STATUS not  be present in  the calling
sequence.

Out of bound errors are the ones most commonly encountered because for
the discrete distributions (binomial, negative binomial, and Poisson),
the cdf  at 0 is  positive not zero.   Hence the inverse cdf  does not
exist for sufficiently small values of the cdf.

\myitem{LOGICAL,  INTENT(IN),  OPTIONAL  ::  CHECK\_INPUT.}  If  this
argument is  present and has the  value .FALSE. then  input values are
not checked for legality.  Generally, this  is a bad idea and can lead
to difficult to find errors.  The argument is included for those cases
in which arguments are checked at a higher level.

\end{description}

{\bf  INPUT AND SEARCH  RANGES.}  Each  input argument  to one  of the
CDF90 routines must  be in the {\em range}  of values.  Values outside
the  range  will be signalled as an error  in the STATUS variable
or cause an abort if STATUS is not present.  {\bf NOTE:} We show the
range as input range for the argument WHICH because WHICH is always
input and never calculated.

For most  distributions (exceptions, the normal and  gamma), finding a
parameter value  producing a specified cdf  value is done  by a search
within  the range  of allowable  values.  If  an answer  is  not found
within this  range, an error is  signalled in the  STATUS variable, or
the program aborts if STATUS is not present in the calling sequence.

Generally,  the  allowable  range  of  the CUM  and  CCUM  arguments  is:
$10^{10}:1-10^{10}$.  Exceptions  occur when one  or both sides
of the support  of the distribution is finite.   Then the search range
includes 0  or 1  or both (depending  on which  end(s) of the  support is
finite).

We somewhat arbitrarily limit the range of values of the various
degrees of freedom arguments to\\
\range{10^{-3}}{10^{10}}
A degrees of freedom argument of zero frequently leads to an
undefined distribution.

The  noncentrality  parameters  of  the noncentral  distributions  are
limited to\\
\range{0,10^4}
The upper  limit is imposed because the cdfs
of  these  distributions are  calculated  as  infinite  series in  the
noncentrality parameter.  The number  of terms to be evaluated becomes
large with this parameter.

\pagebreak

\mysection{cdf\_beta\_mod}

\begin{description}

\item SUBROUTINE CDF\_BETA( WHICH, CUM, CCUM, X, CX, A, B,
STATUS, CHECK\_INPUT)

\item    REAL   (dpkind)   FUNCTION    CUM\_BETA(X,
A, B, STATUS, CHECK\_INPUT)

\item    REAL    (dpkind)FUNCTION    CCUM\_BETA(X,
A, B, STATUS, CHECK\_INPUT)

\item  REAL (dpkind) FUNCTION  INV\_BETA(  CUM, CCUM,
A, B, STATUS, CHECK\_INPUT)

\end{description}

\mysubsection{The Distribution}

The density of the beta distribution is defined on $x$ in $[0,1]$ and
is proportional to
\[ x^a (1-x)^b \]

\mysubsection{Arguments}

\begin{description}

\myitem{INTEGER, INTENT(IN)  :: WHICH.} Integer indicating  which of the
next four arguments is to be calculated.\\
\inrange{1}{4}
\begin{enumerate}
\item CUM and CCUM
\item X and CX
\item A
\item B
\end{enumerate}

\myitem{REAL (dpkind), OPTIONAL :: CUM.} The CDF of the beta distribution.\\
\range{0}{1}

\myitem{REAL (dpkind), OPTIONAL :: CCUM.} One minus the CDF of the
beta distribution.\\
\range{0}{1}

\myitem{REAL (dpkind), OPTIONAL :: X.}  The upper limit of integration of
the beta density.  The lower limit is 0.\\
\range{0}{1}

\myitem{REAL (dpkind),  OPTIONAL :: CX.}   One minus the upper  limit of
integration of the beta density.  The lower limit is 0.\\ \range{0}{1}

\myitem{REAL (dpkind) :: A.}  The first parameter of the beta density.\\
\range{\sprob}{\bdf}

\myitem{REAL (dpkind) :: B.}  The second parameter of the beta density.\\
\range{\sprob}{\bdf}

\myitem{INTEGER, INTENT(OUT) :: STATUS.} Return code.
\begin{description}
\item{-1}  WHICH outside input range
\item{-2}  CUM outside range
\item{-3}  CCUM outside range
\item{-4}  X outside range
\item{-5}  CX outside range
\item{-6}  A outside range
\item{-7}  B outside range
\item{3} CUM + CCUM is not nearly one
\item{4} X + CX is not nearly one
\item{-50} Answer (if any) is below the lower search bound
\item{50} Answer (if any) is above the upper search bound
\end{description}

\myitem{LOGICAL, INTENT(IN), OPTIONAL :: CHECK\_INPUT.}  If PRESENT
and .TRUE. input argument values are not checked for validity.

\end{description}

{\bf NOTE:} CUM and CCUM and also X and CX must add to (nearly) one.

\pagebreak

\mysection{cdf\_binomial\_mod}

\begin{description}

\item SUBROUTINE CDF\_BINOMIAL( WHICH, CUM, CCUM, S, N, PR, CPR,
STATUS, CHECK\_INPUT)

\item    REAL   (dpkind)   FUNCTION    CUM\_BINOMIAL(S,
N, PR, CPR, STATUS, CHECK\_INPUT)

\item    REAL    (dpkind)FUNCTION    CCUM\_BINOMIAL(S,
N, PR, CPR,  STATUS, CHECK\_INPUT)

\item  REAL (dpkind) FUNCTION  INV\_BINOMIAL(  CUM, CCUM,
N, PR, CPR, STATUS, CHECK\_INPUT)

\end{description}

\mysubsection{The Distribution}

The density of the binomial distribution provides the probability of S
successes in N independent trials, each with probability of success
PR.

The density is proportional to
\[ PR^S (1-PR)^{N-S} \]

The  binomial is  extended  to non-integer  values  via the  connection
between the cumulative binomial and the incomplete beta.

\mysubsection{Arguments}

\begin{description}

\myitem{INTEGER, INTENT(IN)  :: WHICH.} Integer indicating  which of the
next four arguments is to be calculated.\\
\inrange{1}{4}
\begin{description}
\item{1} CUM and CCUM
\item{2} S
\item{3} N
\item{4} PR and CPR
\end{description}

\myitem{REAL (dpkind), OPTIONAL :: CUM.} The CDF of the binomial
distribution,
i.e., the probability of 0 to S successes in N trials.\\
\range{0}{\bprob}

\myitem{REAL (dpkind), OPTIONAL :: CCUM.} One minus the CDF of the
binomial distribution.\\
\range{\sprob}{1}

\myitem{REAL (dpkind) :: S.}  The upper limit of summation of
the binomial density.  Note that S must be less than or equal to N.\\
\range{0}{\bbig}

\myitem{REAL (dpkind) :: N.}  The number of independent trials generating
the binomial density. N must be greater than or equal to S.\\
\range{0}{\bbig}

\myitem{REAL (dpkind),  OPTIONAL :: PR.}  The probability  of success in
each independent trial.\\
\range{0}{1}

\myitem{REAL (dpkind),  OPTIONAL :: CPR.}  One minus  the probability of
success in each independent trial;  the probability of failure in each
trial.\\
\range{0}{1}

\myitem{INTEGER, INTENT(OUT) :: STATUS.} Return code.
\begin{description}
\item{-1}  WHICH outside input range
\item{-2}  CUM outside range
\item{-3}  CCUM outside range
\item{-4}  S outside range
\item{-5}  N outside range
\item{-6}  PR outside range
\item{-7}  CPR outside range
\item{3} CUM + CCUM is not nearly one
\item{4} PR + CPR is not nearly one
\item{5} S not between 0 and N.
\item{10} The cumulative binomial is calculated as an incomplete beta.
This error indicates an error in the incomplete beta.  This should not
happen.   
\item{-50} Answer (if any) is below the lower search bound
\item{50} Answer (if any) is above the upper search bound
\end{description}

\myitem{LOGICAL, INTENT(IN), OPTIONAL :: CHECK\_INPUT.}  If PRESENT
and .TRUE. input argument values are not checked for validity.

{\bf NOTE:} CUM and CCUM and also PR and CPR must add to (nearly) one.

\end{description}

\pagebreak

\mysection{cdf\_chisq\_mod}

\begin{description}

\item SUBROUTINE CDF\_CHISQ( WHICH, CUM, CCUM, X, DF,
STATUS, CHECK\_INPUT)

\item    REAL   (dpkind)   FUNCTION    CUM\_CHISQ(X,
DF, STATUS, CHECK\_INPUT)

\item    REAL    (dpkind)FUNCTION    CCUM\_CHISQ(X,
DF, STATUS, CHECK\_INPUT)

\item  REAL (dpkind) FUNCTION  INV\_CHISQ(  CUM, CCUM,
DF, STATUS, CHECK\_INPUT)

\end{description}

\mysubsection{The Distribution}

The chi-squared distribution is the  distribution of the sum of squares
of DF independent unit (mean=0, sd=1) normal deviates.

The density  is defined on $x$ in $[0,\infty)$ and
is proportional to
\[ x^{(DF-2)/2} \exp(-x/2) \]

\mysubsection{Arguments}

\begin{description}

\myitem{INTEGER, INTENT(IN)  :: WHICH.} Integer indicating  which of the
next three arguments is to be calculated.\\
\inrange{1}{3}
\begin{enumerate}
\item CUM and CCUM
\item X
\item DF
\end{enumerate}

\myitem{REAL (dpkind), OPTIONAL :: CUM.} The CDF of the chi-squared
distribution.\\
\range{0}{\bprob}

\myitem{REAL (dpkind), OPTIONAL :: CCUM.} One minus the CDF of the
chi-squared distribution.\\
\range{\sprob}{1}

\myitem{REAL (dpkind) :: X.}  The upper limit of integration of
the chi-squared density.  The lower limit is 0.\\
\range{0}{\immense}

\myitem{REAL (dpkind) :: DF.}  The degrees of freedom of the chi-squared
distribution.\\
\range{\sdf}{\bdf}

\myitem{INTEGER, INTENT(OUT) :: STATUS.} Return code.
\begin{description}
\item{-1}  WHICH outside input range
\item{-2}  CUM outside range
\item{-3}  CCUM outside range
\item{-4}  X outside range
\item{-5}  DF outside range
\item{3} CUM + CCUM is not nearly one
\item{4} X + CX is not nearly one
\item{10} The cumulative chi-squared is computed as an incomplete
beta distribution.  This value indicates an error in the incomplete
beta code.  It really shouldn't happen.
\item{-50} Answer (if any) is below the lower search bound
\item{50} Answer (if any) is above the upper search bound
\end{description}

\myitem{LOGICAL, INTENT(IN), OPTIONAL :: CHECK\_INPUT.}  If PRESENT
and .TRUE. input argument values are not checked for validity.

\end{description}

{\bf NOTE:} CUM and CCUM and also X and CX must add to (nearly) one.

\pagebreak

\mysection{cdf\_f\_mod}

\begin{description}

\item SUBROUTINE CDF\_F( WHICH, CUM, CCUM, F, DFN, DFD,
STATUS, CHECK\_INPUT)

\item    REAL   (dpkind)   FUNCTION    CUM\_F(F,
DFN, DFD, STATUS, CHECK\_INPUT)

\item    REAL    (dpkind)FUNCTION    CCUM\_F(F,
DFN, DFD, STATUS, CHECK\_INPUT)

\item  REAL (dpkind) FUNCTION  INV\_F(  CUM, CCUM,
DFN, DFD, STATUS, CHECK\_INPUT)

\end{description}

\mysubsection{The Distribution}

F  is  the  distribution  of  the  ratio  of  two  independent  random
variables.    The  numerator   random  variable   is   distributed  as
chi-squared  with  DF  degrees   of  freedom  divided  by  DF.   The
denominator  random variable  is distributed  as chi-squared  with DFD
degrees of freedom divided by DFD.

The density of the f distribution is defined on $x$ in $[0,\infty]$ and
is proportional to
\[ \frac{x^{(DFN-2)/2)}}{\left[1+(DFN/DFD)x\right]^{(DFN+DFD)/2}} \]

\mysubsection{Arguments}

\begin{description}

\myitem{INTEGER, INTENT(IN)  :: WHICH.} Integer indicating  which of the
next four arguments is to be calculated.\\
\inrange{1}{2}
\begin{enumerate}
\item CUM and CCUM
\item F
%\item DFN will not be computed because CUM is not monotone in DFN.
%\item DFD will not be computed because CUM is not monotone in DFD.
\end{enumerate}

{\bf  NOTE:} DFN  and DFD  will  not be  computed because  CUM is  not
monotone in either argument.

\myitem{REAL (dpkind), OPTIONAL :: CUM.} The CDF of the f distribution.\\
\range{0}{\bprob}

\myitem{REAL (dpkind), OPTIONAL :: CCUM.} One minus the CDF of the
f distribution.\\
\range{\sprob}{1}

\myitem{REAL (dpkind) :: F.}  The upper limit of integration of
the f density.  The lower limit is 0.\\
\inrange{0}{\immense}

\myitem{REAL (dpkind) :: DFN.}  The numerator degrees of
freedom.\\
\range{\sdf}{\bdf}

\myitem{REAL (dpkind) :: DFD.}  The denominator degrees of freedom.\\
\range{\sdf}{\bdf}

\myitem{INTEGER, INTENT(OUT) :: STATUS.} Return code.
\begin{description}
\item{-1}  WHICH outside input range
\item{-2}  CUM outside range
\item{-3}  CCUM outside range
\item{-4}  F outside range
\item{-5}  DFN outside range
\item{-6}  DFD outside range
\item{3} CUM + CCUM is not nearly one
\item{10} The cumulative F is computed as an incomplete
beta distribution.  This value indicates an error in the incomplete
beta code.  It really shouldn't happen.
\item{-50} Answer (if any) is below the lower search bound
\item{50} Answer (if any) is above the upper search bound
\end{description}

\myitem{LOGICAL, INTENT(IN), OPTIONAL :: CHECK\_INPUT.}  If PRESENT
and .TRUE. input argument values are not checked for validity.

\end{description}

{\bf NOTE:} CUM and CCUM  must add to (nearly) one.

{\bf  NOTE:}  The  value  of  the  CDF of the  f  distribution  is  not
necessarily monotone in either  degree of freedom argument.  There may
thus  be two  values that  provide a  given DCF  value.   This routine
assumes monotonicity and will find an arbitrary one of the two values.

\pagebreak

\mysection{cdf\_gamma\_mod}

\begin{description}

\item SUBROUTINE CDF\_GAMMA( WHICH, CUM, CCUM, X, SHAPE, SCALE,
STATUS, CHECK\_INPUT)

\item    REAL   (dpkind)   FUNCTION    CUM\_GAMMA(X,
SHAPE, SCALE, STATUS, CHECK\_INPUT)

\item    REAL    (dpkind)FUNCTION    CCUM\_GAMMA(X,
SHAPE, SCALE, STATUS, CHECK\_INPUT)

\item  REAL (dpkind) FUNCTION  INV\_GAMMA(  CUM, CCUM,
SHAPE, SCALE, STATUS, CHECK\_INPUT)

\end{description}

\mysubsection{The Distribution}

The density of the GAMMA distribution is proportional to:

\[ (x/SCALE)^{SHAPE-1} \exp(-x/SCALE) \]

\mysubsection{Arguments}

\begin{description}

\myitem{INTEGER, INTENT(IN)  :: WHICH.} Integer indicating  which of the
next four arguments is to be calculated.\\
\inrange{1}{4}
\begin{enumerate}
\item CUM and CCUM
\item X
\item SHAPE
\item SCALE
\end{enumerate}

\myitem{REAL (dpkind), OPTIONAL :: CUM.} The CDF of the gamma distribution.\\
\range{0}{\bprob}

\myitem{REAL (dpkind), OPTIONAL :: CCUM.} One minus the CDF of the
gamma distribution.\\
\range{\sprob}{1}

\myitem{REAL (dpkind) :: X.}  The upper limit of integration of
the gamma density.  The lower limit is 0.\\
\range{0}{\immense}

\myitem{REAL (dpkind) :: SHAPE.}  The shape parameter of the
distribution.\\
\range{\sprob}{\immense}

\myitem{REAL (dpkind) :: SCALE.}  The scale parameter of the distribution.\\
\range{\sprob}{\immense}

\myitem{INTEGER, INTENT(OUT) :: STATUS.} Return code.
\begin{description}
\item{-1}  WHICH outside input range
\item{-2}  CUM outside range
\item{-3}  CCUM outside range
\item{-4}  X outside range
\item{-5}  SHAPE outside range
\item{-6}  SCALE outside range
\item{3} CUM + CCUM is not nearly one
\item{5} Some error in inverse gamma routine
\end{description}

\myitem{LOGICAL, INTENT(IN), OPTIONAL :: CHECK\_INPUT.}  If PRESENT
and .TRUE. input argument values are not checked for validity.

\end{description}

{\bf NOTE:} CUM and CCUM  must add to (nearly) one.

\pagebreak

\mysection{cdf\_neg\_binomial\_mod}

\begin{description}

\item SUBROUTINE CDF\_NEG\_BINOMIAL( WHICH, CUM, CCUM, F, S, PR, CPR,
STATUS, CHECK\_INPUT)

\item    REAL   (dpkind)   FUNCTION    CUM\_NEG\_BINOMIAL(F, S,
N, PR, CPR, STATUS, CHECK\_INPUT)

\item    REAL    (dpkind)FUNCTION    CCUM\_NEG\_BINOMIAL(F, S,
N, PR, CPR,  STATUS, CHECK\_INPUT)

\item  REAL (dpkind) FUNCTION  INV\_NEG\_BINOMIAL(  CUM, CCUM,
N, PR, CPR, STATUS, CHECK\_INPUT)

\end{description}

\mysubsection{The Distribution}

The  density  of  the  negative  binomial  distribution  provides  the
probability  of  precisely  F  failures  before the  S'th  success  in
independent binomial trials, each with probability of success PR.

The density is

\[ \left( \begin{array}{c} F+S-1\\ S-1 \end{array} \right) PR^S (1-PR)^F \]

The cumulative distribution function is  the probability of F or fewer
failures before the F'th success.

The negative  binomial is extended to  non-integer values of  F via the
relation between the cumulative  distribution function of the negative
binomial and the incomplete beta function.

\mysubsection{Arguments}

\begin{description}

\myitem{INTEGER, INTENT(IN)  :: WHICH.} Integer indicating  which of the
next four arguments is to be calculated.\\
\inrange{1}{4}
\begin{description}
\item{1} CUM and CCUM
\item{2} F
\item{3} S
\item{4} PR and CPR
\end{description}

\myitem{REAL (dpkind), OPTIONAL :: CUM.} The CDF of the negative-binomial
distribution.\\
\range{0}{\bprob}

\myitem{REAL (dpkind), OPTIONAL }{: CCUM.} One minus the CDF of the
binomial distribution.\\
\range{\sprob}{1}

\myitem{REAL (dpkind) :: F.}  The number of failures before the S'th success.\\
\range{0}{\bbig}

\myitem{REAL (dpkind) :: S.}  The number of successes to occur.\\
\inrange{0}{\bbig}

\myitem{REAL  (dpkind)  :: PR.}   The  probability  of  success in  each
independent trial.\\
\range{0}{1}

\myitem{REAL (dpkind) :: CPR.}  One  minus the probability of success in
each  independent trial;  the probability  of failure  in  each trial.\\
\range{0}{1}

\myitem{INTEGER, INTENT(OUT) :: STATUS.} Return code.
\begin{description}
\item{-1}  WHICH outside input range
\item{-2}  CUM outside range
\item{-3}  CCUM outside range
\item{-4}  F outside range
\item{-5}  S outside range
\item{-6}  PR outside range
\item{-7}  CPR outside range
\item{3} CUM + CCUM is not nearly one
\item{4} PR + CPR is not nearly one
\item{10} The cumulative negative binomial is computed as an
  incomplete beta.  This value of STATUS indicates an error in the
  incomplete beta routine.  It really shouldn't happen.
\item{-50} Answer (if any) is below the lower search bound
\item{50} Answer (if any) is above the upper search bound
\end{description}

\myitem{LOGICAL, INTENT(IN), OPTIONAL :: CHECK\_INPUT.}  If PRESENT
and .TRUE. input argument values are not checked for validity.

\end{description}

{\bf NOTE:} CUM and CCUM and also PR and CPR must add to (nearly) one.

\pagebreak

\mysection{cdf\_nc\_chisq\_mod}

\begin{description}

\item SUBROUTINE CDF\_NC\_CHISQ( WHICH, CUM, CCUM, X, DF, PNONC,
STATUS, CHECK\_INPUT)

\item    REAL   (dpkind)   FUNCTION    CUM\_NC\_CHISQ(X,
DF, PNONC, STATUS, CHECK\_INPUT)

\item    REAL    (dpkind) FUNCTION CCUM\_NC\_CHISQ(X,
DF, PNONC, STATUS, CHECK\_INPUT)

\item  REAL (dpkind) FUNCTION  INV\_NC\_CHISQ(  CUM, CCUM,
DF, PNONC, STATUS, CHECK\_INPUT)

\end{description}

\mysubsection{The Distribution}

The noncentral  chi-squared distribution is the sum  of DF independent
normal  distributions  with  unit  standard deviations,  but  possibly
non-zero  means .  Let  the mean  of the  $i$th normal  be $\delta_i$.
Then PNONC = $ \sum_i \delta_i$.

\mysubsection{Arguments}

\begin{description}

\myitem{INTEGER, INTENT(IN)  :: WHICH.} Integer indicating  which of the
next three arguments is to be calculated.\\
\begin{enumerate}
\item CUM and CCUM
\item DF
\item PNONC
\end{enumerate}

\myitem{REAL (dpkind), OPTIONAL :: CUM.} The CDF of the noncentral
chi-squared  distribution.\\
\range{0}{\bprob}

\myitem{REAL (dpkind), OPTIONAL :: CCUM.} One minus the CDF of the
noncentral chi-squared distribution.\\
\range{\sprob}{1}

\myitem{REAL (dpkind) :: X.} The upper limit of integration of the
noncentral
chi-squared distribution.  (The lower limit is 0.)
\range{0}{\immense}

\myitem{REAL (dpkind) :: DF.}  The  degrees of
freedom of the noncentral chi-squared distribution.\\
\inrange{\sdf}{\bdf}

\myitem{REAL (dpkind) :: PNONC.}  The noncentrality parameter.\\
\range{\spnonc}{\bpnonc}

\myitem{INTEGER, INTENT(OUT) :: STATUS.} Return code.
\begin{description}
\item{-1}  WHICH outside input range
\item{-2}  CUM outside range
\item{-3}  CCUM outside range
\item{-4}  X outside range
\item{-5}  DF outside range
\item{-6}  PNONC outside range
\item{3} CUM + CCUM is not nearly one
\item{-50} Answer (if any) is below the lower search bound
\item{50} Answer (if any) is above the upper search bound
\end{description}

\myitem{LOGICAL, INTENT(IN), OPTIONAL :: CHECK\_INPUT.}  If PRESENT
and .TRUE. input argument values are not checked for validity.

\end{description}

{\bf NOTE:} CUM and CCUM  must add to (nearly) one.

\pagebreak

\mysection{cdf\_nc\_f\_mod}

\begin{description}

\item SUBROUTINE CDF\_F( WHICH, CUM, CCUM, F, DFN, DFD, PNONC,
STATUS, CHECK\_INPUT)

\item    REAL   (dpkind)   FUNCTION    CUM\_F(F,
DFN, DFD, PNONC, STATUS, CHECK\_INPUT)

\item    REAL    (dpkind)FUNCTION    CCUM\_F(F,
DFN, DFD, PNONC, STATUS, CHECK\_INPUT)

\item  REAL (dpkind) FUNCTION  INV\_F(  CUM, CCUM,
DFN, DFD, PNONC, STATUS, CHECK\_INPUT)

\end{description}

\mysubsection{The Distribution}

The noncentral F  is the distribution of the  ratio of two independent
random variables.   The numerator random variable is  distributed as a
noncentral chi-squared  with DFN  degrees of freedom  and noncentrality
parameter  PNONC divided by  DFN.  The  denominator random  variable is
distributed  as a (central)  chi-squared with  DFD degrees  of freedom
divided by DFD.

\mysubsection{Arguments}

\begin{description}

\myitem{INTEGER, INTENT(IN)  :: WHICH.} Integer indicating  which of the
next five arguments is to be calculated.\\
\inrange{1}{2}
\begin{enumerate}
\item CUM and CCUM
\item F
\end{enumerate}

{\bf  NOTE:} DFN  and DFD  will  not be  computed because  CUM is  not
monotone in either argument.

\myitem{REAL  (dpkind), OPTIONAL ::  CUM.} The  CDF of  the  noncentral f
distribution.\\ \range{0}{\bprob}

\myitem{REAL (dpkind), OPTIONAL :: CCUM.} One minus the CDF of the
noncentral f distribution.\\
\range{\sprob}{1}

\myitem{REAL (dpkind) :: F.}  The upper limit of integration of
the noncentral f density.  The lower limit is 0.\\
\range{0}{\immense}

\myitem{REAL (dpkind) :: DFN.}  The numerator degrees of
freedom.\\
\range{\sdf}{\bdf}

\myitem{REAL (dpkind) :: DFD.}  The denominator degrees of freedom.\\
\range{\sdf}{\bdf}

\myitem{REAL (dpkind) :: PNONC.}  The noncentrality parameter.\\
\range{\spnonc}{\bpnonc}

\myitem{INTEGER, INTENT(OUT) :: STATUS.} Return code.
\begin{description}
\item{-1}  WHICH outside input range
\item{-2}  CUM outside range
\item{-3}  CCUM outside range
\item{-4}  F outside range
\item{-5}  DFN outside range
\item{-6}  DFD outside range
\item{3} CUM + CCUM is not nearly one
\item{-50} Answer (if any) is below the lower search bound
\item{50} Answer (if any) is above the upper search bound
\end{description}

\myitem{LOGICAL, INTENT(IN), OPTIONAL :: CHECK\_INPUT.}  If PRESENT
and .TRUE. input argument values are not checked for validity.

\end{description}

{\bf NOTE:} CUM and CCUM  must add to (nearly) one.

\pagebreak

\mysection{cdf\_nc\_t\_mod}

\begin{description}

\item SUBROUTINE CDF\_T( WHICH, CUM, CCUM, T, DF, PNONC,
STATUS, CHECK\_INPUT)

\item    REAL   (dpkind)   FUNCTION    CUM\_T(T,
DF, PNONC, STATUS, CHECK\_INPUT)

\item    REAL    (dpkind)FUNCTION    CCUM\_T(T,
DF, PNONC, STATUS, CHECK\_INPUT)

\item  REAL (dpkind) FUNCTION  INV\_T(  CUM, CCUM,
DF, PNONC, STATUS, CHECK\_INPUT)

\end{description}

\mysubsection{The Distribution}

The noncentral T  is the distribution of the  ratio of two independent
random variables.   The numerator random variable is  distributed as a
normal distribution  with mean PNONC and variance  1.  The denominator
random variable  is distributed  as a the  square root of  a (central)
chi-squared with DF degrees of freedom divided by DF.

\mysubsection{Arguments}

\begin{description}

\myitem{INTEGER, INTENT(IN)  :: WHICH.} Integer indicating  which of the
next arguments is to be calculated.\\
\inrange{1}{4}
\begin{enumerate}
\item CUM and CCUM
\item T
\item DF
\item PNONC
\end{enumerate}

\myitem{REAL  (dpkind), OPTIONAL ::  CUM.} The  CDF of  the  noncentral t
distribution.\\
\range{\sprob}{\bprob}

\myitem{REAL (dpkind), OPTIONAL :: CCUM.} One minus the CDF of the
noncentral t distribution.\\
\range{\sprob}{\bprob}

\myitem{REAL (dpkind)  :: T.}  The upper  limit of integration  of the
noncentral t
density.     The    lower    limit    is    $-\infty$.\\
\range{-\immense}{\immense}

\myitem{REAL (dpkind) :: DF.}  The degrees of
freedom.\\
\range{\sdf}{\bdf}

\myitem{REAL (dpkind) :: PNONC.}  The noncentrality parameter.\\
\range{\spnonc}{\bpnonc}

\myitem{INTEGER, INTENT(OUT) :: STATUS.} Return code.
\begin{description}
\item{-1}  WHICH outside input range
\item{-2}  CUM outside range
\item{-3}  CCUM outside range
\item{-4}  T outside range
\item{-5}  DF outside range
\item{-6}  PNONC outside range
\item{3} CUM + CCUM is not nearly one
\item{10} The noncentral t calculation uses the central t calculation
which is reduced to the calculation of an incomplete beta.  This value
of STATUS indicates an error in the incomplete beta calculation.  It
really shouldn't happen.
\item{-50} Answer (if any) is below the lower search bound
\item{50} Answer (if any) is above the upper search bound
\end{description}

\myitem{LOGICAL, INTENT(IN), OPTIONAL :: CHECK\_INPUT.}  If PRESENT
and .TRUE. input argument values are not checked for validity.

\end{description}

\pagebreak

\mysection{cdf\_normal\_mod}

\begin{description}

\item SUBROUTINE CDF\_NORMAL( WHICH, CUM, CCUM, X, MEAN, SD,
STATUS, CHECK\_INPUT )

\item REAL (dpkind) FUNCTION CUM\_NORMAL(X, MEAN, SD,
 STATUS, CHECK\_INPUT)

\item REAL (dpkind) FUNCTION CCUM\_NORMAL(X, MEAN, SD, STATUS,
CHECK\_INPUT)

\item REAL (dpkind) FUNCTION INV\_NORMAL(CUM, CCUM, MEAN, SD, STATUS,
CHECK\_INPUT)

\end{description}

\mysubsection{The Distribution}

The density of the normal distribution is proportional to

\[ \exp\left(-\frac{(X-MEAN)^2}{2 SD^2}\right) \]

\mysubsection{Arguments}

\begin{description}

\myitem{INTEGER, INTENT(IN)  :: WHICH.} Integer indicating  which of the
next arguments is to be calculated.\\
\inrange{1}{4}
\begin{enumerate}
\item CUM and CCUM
\item X
\item MEAN
\item SD
\end{enumerate}

\myitem{REAL  (dpkind), OPTIONAL ::  CUM.} The  CDF of  the  normal
distribution.\\
\range{\sprob}{\bprob}

\myitem{REAL (dpkind), OPTIONAL :: CCUM.} One minus the CDF of the
normal  distribution.\\
\range{\sprob}{\bprob}

\myitem{REAL  (dpkind) ::  X.}  The  upper limit  of integration  of the
normal     density.     The     lower     limit    is     $-\infty$.\\
\range{-\immense}{\immense}

\myitem{REAL (dpkind), OPTIONAL:: MEAN.} The mean of the normal
distribution.  If omitted, the value 0 is used.\\
\range{-\immense}{\immense}

\myitem{REAL (dpkind), OPTIONAL:: SD.} The standard deviation of the normal
distribution.  If omitted, the value 1 is used.\\
\range{\sprob}{\immense}

\myitem{INTEGER, INTENT(OUT), OPTIONAL :: STATUS.} Return code.
\begin{description}
\item{-1}  WHICH outside input range
\item{-2}  CUM outside range
\item{-3}  CCUM outside range
\item{-4}  X outside range
\item{-5}  MEAN outside range
\item{-6}  SD outside range
\item{3} CUM + CCUM is not nearly one
\end{description}

\myitem{LOGICAL, INTENT(IN), OPTIONAL :: CHECK\_INPUT.}  If PRESENT
and .TRUE. input argument values are not checked for validity.

\end{description}

\pagebreak

\mysection{cdf\_poisson\_mod}

\begin{description}

\item SUBROUTINE CDF\_POISSON( WHICH, CUM, CCUM, S, LAMBDA, STATUS,
    CHECK\_INPUT)

\item REAL (dpkind) FUNCTION
CUM\_POISSON(S, LAMBDA, STATUS, CHECK\_INPUT)

\item REAL (dpkind) FUNCTION CCUM\_POISSON(S, LAMBDA, STATUS,
CHECK\_INPUT)

\item FUNCTION INV\_POISSON( CUM, CCUM, LAMBDA, STATUS, CHECK\_INPUT)

\end{description}

\mysubsection{The Distribution}

The density of the Poisson distribution (probability of observing S
events) is:

\[ \frac{LAMBDA^S}{S!} \exp(-LAMBDA) \]

The Poisson distribution  is extended to non-integer values  of S using
the  relation  between the  cumulative  distribution  function of  the
Poisson distribution and the gamma distribution.

\mysubsection{Arguments}

\begin{description}

\myitem{INTEGER, INTENT(IN)  :: WHICH.} Integer indicating  which of the
next arguments is to be calculated.\\
\inrange{1}{3}
\begin{enumerate}
\item CUM and CCUM
\item S
\item LAMBDA
\end{enumerate}

\myitem{REAL  (dpkind), OPTIONAL ::  CUM.} The  CDF of  the Poisson
distribution.\\
\range{0}{\bprob}

\myitem{REAL (dpkind), OPTIONAL :: CCUM.} One minus the CDF of the
Poisson  distribution.\\
\range{\sprob}{1}

\myitem{REAL  (dpkind) ::  S.}  The  upper limit  of summation  of the
Poisson density.     The     lower     limit    is     $0$.\\
\range{0}{\immense}

\myitem{REAL (dpkind) :: LAMBDA.} The mean of the Poisson
distribution.
\range{\sprob}{\immense}

\myitem{INTEGER, INTENT(OUT), OPTIONAL :: STATUS.} Return code.
\begin{description}
\item{-1}  WHICH outside input range
\item{-2}  CUM outside range
\item{-3}  CCUM outside range
\item{-4}  X outside range
\item{-5}  MEAN outside range
\item{-6}  SD outside range
\item{3} CUM + CCUM is not nearly one
\item{10} The cumulative Poisson is calculated as an incomplete
gamma distribution.  This status indicates that an error occurred
in the incomplete gamma code.  It really shouldn't happen.
\item{-50} Answer (if any) is below lower bound on range
\item{50} Answer (if any) is above upper bound on range
\end{description}


\myitem{LOGICAL, INTENT(IN), OPTIONAL :: CHECK\_INPUT.}  If PRESENT
and .TRUE. input argument values are not checked for validity.

\end{description}

\pagebreak

\mysection{cdf\_t\_mod}

\begin{description}

\item SUBROUTINE CDF\_T( WHICH, CUM, CCUM, T, DF,
                        STATUS, CHECK\_INPUT )

\item REAL (dpkind) FUNCTION CUM\_T(T, DF, STATUS, CHECK\_INPUT)

\item REAL (dpkind)  FUNCTION CCUM\_T(T, DF, STATUS, CHECK\_INPUT)

\item REAL (dpkind)  FUNCTION INV\_T(CUM, CCUM, T, DF, STATUS,
CHECK\_INPUT)

\end{description}

\mysubsection{The Distribution}

The density is proportional to

\[ \left[ 1 + \frac{T^2}{DF} \right]^{(DF+1)/2} \]

\mysubsection{Arguments}

\begin{description}

\myitem{INTEGER, INTENT(IN)  :: WHICH.} Integer indicating  which of the
next arguments is to be calculated.\\
\inrange{1}{3}
\begin{enumerate}
\item CUM and CCUM
\item T
\item DF
\end{enumerate}

\myitem{REAL  (dpkind), OPTIONAL ::  CUM.} The  CDF of  the  noncentral t
distribution.\\
\range{\sprob}{\bprob}

\myitem{REAL (dpkind), OPTIONAL :: CCUM.} One minus the CDF of the
noncentral t distribution.\\
\range{\sprob}{\bprob}

\myitem{REAL (dpkind)  :: T.}  The upper  limit of integration  of the
noncentral t
density.     The    lower    limit    is    $-\infty$.\\
\range{-\immense}{\immense}

\myitem{REAL (dpkind) :: DF.}  The degrees of
freedom.\\
\range{\sdf}{\bdf}

\myitem{INTEGER, INTENT(OUT) :: STATUS.} Return code.
\begin{description}
\item{-1}  WHICH outside input range
\item{-2}  CUM outside range
\item{-3}  CCUM outside range
\item{-4}  T outside range
\item{-5}  DF outside range
\item{3} CUM + CCUM is not nearly one
\item{10}  The cumulative t is computed as an incomplete
beta distribution.  This value indicates an error in the incomplete
beta code.  It really shouldn't happen.
\item{-50} Answer (if any) is below the lower search bound
\item{50} Answer (if any) is above the upper search bound
\end{description}

\myitem{LOGICAL, INTENT(IN), OPTIONAL :: CHECK\_INPUT.}  If PRESENT
and .TRUE. input argument values are not checked for validity.

\end{description}

\end{document}
