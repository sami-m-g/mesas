% Uncomment next line for pdflatex
%\documentclass[12pt,pdftex]{article}

% Comment next line for pdftex
\documentclass[12pt,dvips]{article}

\usepackage{xspace}
\usepackage{color}
%uncomment next two lines for pdflatex
%\usepackage[bookmarks,bookmarksopen,
%    pdftex=true]{hyperref}

\definecolor{Orange}{cmyk}{0,0.61,0.87,0}

\addtolength{\oddsidemargin}{-0.5in}
\addtolength{\textheight}{1.0in}
\addtolength{\textwidth}{0.5in}
\setlength{\parindent}{0in}
\setlength{\parskip}{0.1in}
\newcommand{\meta}[1]{{\ensuremath{<}}#1{\ensuremath{{>}}}}
\newcommand{\us}{\underline{ }}
\newcommand{\po}{\ensuremath{p_{h0}}\xspace}
\newcommand{\thet}{\ensuremath{\theta}}

\newcommand{\mysection}[1]
    {\color{blue} \section{#1} \color{black}}
\newcommand{\mysubsection}[1]
    {\color{green}\subsection{#1}\color{black}}
\newcommand{\mysubsubsection}[1]
    {\color{Orange}\subsubsection{#1}\color{black}}
\newcommand{\vs}{\vspace*{\fill}}
\newcommand{\CR}{\ensuremath{>} \meta{CR}\\}
\newcommand{\Input}[1]{\ensuremath{>}{\ttfamily #1\\}}
\newenvironment{commentary}{\begin{quote} \color{Orange}
\begin{center} Begin Commentary\\ \vspace{0.1in} 
\end{center} \color{black} }
{\color{Orange} \begin{center} End Commentary\\ \end{center} \color{black}
      \end{quote} \normalsize }
\newcommand{\commentaryhead}[1]
{\begin{center} \large \color{green} #1\\ 
    \normalsize \color{black} \end{center} }

\begin{document}

\Large \bfseries
     
{\centering \color{blue}     
                                    STATTAB\\
                          Version 2.0: March, 2002\\
\vspace{0.1in}


     Calculates Cumulative Distribution Functions,\\
               Inverses, and Parameters of Distributions.\\

\vspace{0.1in}

}



{\centering \color{green}
                  Barry W. Brown  \hfill             David Gutierrez\\
                  James Lovato    \hfill            Dan Serachitopol\\
                  Marty Spears    \hfill            John Venier\\}

\vs

     This  work supported  by   grant CA-16672   from the  National  Cancer
     Institute.\\

  Copyright 1989-2002 for:
\begin{quote}            
         The University of Texas, M.D. Anderson Cancer Center
                    Department of Biomathematics, Box 237
                    1515 Holcombe Boulevard
                    Houston, TX      77030
\end{quote}
     Contact: BWB at above address or bwb@mdanderson.org

     This program may be freely copied and (noncommercially) distributed.

     This program can be obtained from\\
{\centering http://odin.mdacc.tmc.edu/anonftp/}

\normalsize \normalfont

\vs

\pagebreak

\mysection{Legalities}

We   want our  computer  programs to  be  as  widely used as possible,
consequently    unrestricted  use   of  them  within     a computer is
permissible.    Use   within commercial  enterprises   for  commercial
purposes  is explicitly allowed   as    is academic, research,     and
government or any other use.  

Copies of any   form of  the   code (source  or  executable)  and  the
documentation can  be made and   redistributed  freely.  You may  give
copies  to students, coworkers, friends,  and  enemies.  We do require
that   the  copyright   and  LEGALITIES  section   of   the code   and
documentation be preserved in  such copies.

Modifications   and improvements to the  code  can similarly be freely
copied and  distributed.  Modified code be clearly   marked as such so
that we are not  requested to fix it.   We would appreciate a copy  of
any improvements   for   incorporation into  later   versions  of  our
programs.

Inclusion of  our code in   software archives and its distribution  as
freeware is encouraged.

Incorporation into Commercial Packages

Incorporation of our code into a commercial package  that will be sold
requires   written permission  from  an administrative  officer of the
Department of Biomathematics.   Our  current policy is to  grant  such
permission without payment providing:

(1)  We are informed  of and approve  the nature of  the product being
developed or prepared for   distribution.  

(2) Appropriate  acknowledgment    of  our   code   is   made   in the
documentation accompanying the product.

This program  contains code from  the following publications, and code
from ACM publications is subject to the ACM policy (below).

\mysubsection{References}

\mysubsubsection{Incomplete Beta}

DiDinato, A. R. and Morris,  A.   H. (1993)  ``Algorithm 708: Significant
Digit Computation of the Incomplete  Beta  Function Ratios.''  {\em ACM
Trans. Math.  Softw. 18}, 360-373.

\mysubsubsection{Incomplete Gamma}

DiDinato, A. R. and Morris, A. H. (1986) ``Computation of the  incomplete
gamma function  ratios  and their  inverse.''   {\em ACM  Trans.  Math.
Softw. 12}, 377-393.

\mysubsubsection{Cumulative Normal}

Cody, W.D. (1993). ``ALGORITHM 715: SPECFUN - A Portable FORTRAN
Package of Special Function Routines and Test Drivers''
{\em ACM Trans. Math. Softw. 19}, 22-32.

\mysubsubsection{Inverse Normal}

``Algorithm AS241'' (1988) {\em Appl. Statist. 37}, NO. 3, 477-484.

\mysubsubsection{Finding a Zero of a Monotone Function}

Alefeld,  G.  E.,  Potra,  F.  A., Shi,  Y.  (1995)  ``Algorithm  748:
Enclosing  Zeros  of  Continuous   Functions.'',  by  G.  E.  Alefeld,
F.  A. Potra, YiXun  Shi, {\em ACM  Trans.  Math. Softw.,
21}, No. 3, 327-344

\mysubsection{ACM Policy on Use of Code}

Here  is the  software Policy of  the  ACM.

\begin{quote}

     Submittal of  an  algorithm    for publication  in   one of   the  ACM
     Transactions implies that unrestricted use  of the algorithm within  a
     computer is permissible.   General permission  to copy and  distribute
     the algorithm without fee is granted provided that the copies  are not
     made  or   distributed for  direct   commercial  advantage.    The ACM
     copyright notice and the title of the publication and its date appear,
     and  notice is given that copying  is by permission of the Association
     for Computing Machinery.  To copy otherwise, or to republish, requires
     a fee and/or specific permission.

     Krogh, F.  (1997) ``Algorithms  Policy.''  {\em ACM  Tran.  Math.
     Softw.  13}, 183-186.

\end{quote}

We do not know the policy  of the Royal Statistical Society; they have
discontinued publishing  algorithms.  However,  they made a  number of
these  programs available  on Statlib  on condition  that there  be no
charge for their distribution.

Here is our standard disclaimer.

\begin{quote}

{\centering NO WARRANTY\\}

WE PROVIDE ABSOLUTELY  NO WARRANTY  OF ANY  KIND  EITHER  EXPRESSED OR
IMPLIED,  INCLUDING BUT   NOT LIMITED TO,  THE  IMPLIED  WARRANTIES OF
MERCHANTABILITY AND FITNESS FOR A PARTICULAR PURPOSE.  THE ENTIRE RISK
AS TO THE QUALITY AND PERFORMANCE OF THE PROGRAM IS  WITH YOU.  SHOULD
THIS PROGRAM PROVE  DEFECTIVE, YOU ASSUME  THE COST  OF  ALL NECESSARY
SERVICING, REPAIR OR CORRECTION.

IN NO  EVENT  SHALL THE UNIVERSITY  OF TEXAS OR  ANY  OF ITS COMPONENT
INSTITUTIONS INCLUDING M. D.   ANDERSON HOSPITAL BE LIABLE  TO YOU FOR
DAMAGES, INCLUDING ANY  LOST PROFITS, LOST MONIES,   OR OTHER SPECIAL,
INCIDENTAL   OR  CONSEQUENTIAL DAMAGES   ARISING   OUT  OF  THE USE OR
INABILITY TO USE (INCLUDING BUT NOT LIMITED TO LOSS OF DATA OR DATA OR
ITS ANALYSIS BEING  RENDERED INACCURATE OR  LOSSES SUSTAINED  BY THIRD
PARTIES) THE PROGRAM.

(Above NO WARRANTY modified from the GNU NO WARRANTY statement.)
\end{quote}

\mysubsection{Thanks to Our Supporters}
     
The creation  of this  code was  supported in part  by the  Core Grant
CA11672 of  the National  Cancer Institute of  NIH to M.   D. Anderson
Hospital.  Other  support was provided by  the State of  Texas and the
Robert R. Herring Professorship in Cancer Research held by Dr.  Brown.
     
\pagebreak

\mysection{Introduction}

     STATTAB    calculates    cumulative functions,  their   inverses,  and
     parameters of the following distributions:

\begin{enumerate}
\item{Incomplete Beta}
\item{Binomial}
\item{Negative Binomial}
\item{Chi-square}
\item{Non-central chi-square}
\item{Variance Ratio - F}
\item{Non-central F}
\item{Incomplete Gamma}
\item{Normal}
\item{Poisson}
\item{T}
\item{Non-central T}
\end{enumerate}
     
     When normal or T cdfs are being  computed, STATTAB also
     calculates  the associated two-sided  p-value.  When  chi-square  or F
     cdfs  are calculated,  the associated  many sided   p-values are  also
     printed.
     
     The probability of exactly n events is calculated and printed with the
     cumulative distribution for the binomial and Poisson distributions.
     
     Calculations  can be  performed  on a list  of  values  for  producing
     customized tables used for power or other calculations.  A report file
     may be specified to contain all answers calculated by STATTAB.
     
\mysubsection{ Notes on the Calculations}
     
     The  cumulative distribution functions   for the binomial  and Poisson
     distributions  are   calculated at  non-integral  arguments  using the
     relation between the cumulative binomial and the beta distribution and
     between the Poisson and the chi-square.
     
     The  individual terms for the  binomial  and  Poisson distribution are
     calculated for  integer values obtained  by truncating   the arguments
     given.
     
\mysubsection{Notes on the Use of the Program}

     STATTAB  allows the user to  choose a distribution from a menu.
     It  then prompts the user for  all the parameters  needed.  
     
\begin{enumerate}

\item NOTE: Within a distribution specific menu, you can type 'help'
  to obtain both a description of the distribution and of the
  parameters that must be entered.

\item There are several valid  methods of entering values for parameters.
     These include an integer,  a decimal number specified  by X.X (where X
     is an integer)  or an exponential  number specified by X.XeNN, X.XdNN,
     or  XeNN.   (The  letters  'e' and  'd'  are  interchangeable in  this
     context.  In these  forms, NN  is  an integer,  with  a minus sign  if
     necessary.  X.XeNN means X.X multiplied by 10  raised to the power NN.
     Thus, 0.0123 and 123e-4 represent the same number.

     Many machines   including  most  Unix   machines, PCs   and  Macintosh
     computers use IEEE  representation.  This   means that the    smallest
     absolute magnitudes   that can be   handled are approximately -1e-308,
     +1e-308.  Numbers smaller than this are set to 0.  The largest numbers
     in absolute magnitude are approximately -1e308, 1e308.  Numbers larger
     than this cause some abnormal condition, perhaps a program crash.

\item Some of the parameters are complementary parameters,  i.e. they sum
     to one.  For  example,  C , the  cumulative  probablity, and  1-C  are
     complementary  parameters.  Enter only  ONE  of the  two complementary
     parameters and enter a '.' to skip the other.

     All  complementary  parameters are specified in the form  X  and  1-X.
     Complementary  parameters allow for  more accurate computations of the
     parameter  X in the case that X is  close to one.  This is because the
     computer  will  "round off"  numbers  close  to one.  For example,  if
     X = 1.0 - 1.0e-50,   the   computer  will   store   X = 1.0   so  that
     calculations involving  X  can not be done accurately.  However, if we
     define  1-X = 1.0e-50,  we  can  use  this  value  and  the  fact that 
     X + 1-X = 1 to make accurate calculations.

\item The user must always enter EXACTLY one '?'.  The '?' indicates that
     the corresponding parameter value is to  be calculated from the values
     of the other parameters.
     
\item The user can enter one 'T' for a parameter which allows the user to
     specify a list of values.  The  answer is  calculated for each item on
     the list.
     
\item Either commas or spaces can be used to separate values.
     
\item  Parameters whose value is  to remain  the same as  in the previous
     calculation are indicated by the sign, "=".
    
\end{enumerate}


\mysection{ AN ANNOTATED RUN}    
 
\begin{verbatim}




                                    STATTAB
                          Version 2.0: March, 2002

     Calculates Statistical Cumulative Distribution Functions  (P-Values),
              Their Inverses, and Parameters of the Distributions.

                  Barry W. Brown              David Gutierrez
                  James Lovato                Dan Serachitopol
                  Marty Spears                John Venier

     This  work supported  by   grant CA-16672   from the  National  Cancer
     Institute.  Copyright 1989-2002 for:
                    The University of Texas, M.D. Anderson Cancer Center
                    Department of Biomathematics, Box 237
                    1515 Holcombe Boulevard
                    Houston, TX      77030
     Contact: BWB at above address or bwb@mdanderson.org

     This program may be freely copied and (noncommercially) distributed.

     This program can be obtained from
                    http://odin.mdacc.tmc.edu/anonftp/



 Press the Return or Enter key to continue ...

\end{verbatim}

\begin{commentary}

A REPORT file contains your input plus the results of calculations
made.  If you want to refer to the results of the program after you
have finished running it, then you want a REPORT file.

The name of the report file must be a legal file name on your computer
system. 

\end{commentary}

\begin{verbatim}

Do you want a REPORT file (y/n)?
 Please enter one of [yn]: > y

     Enter the  name  of the report file

 > bwb,rep

 Enter 0 to quit or  the number of a distribution.  Distribution number
 then 'HELP' provides info on that distribution.


  0 - EXIT Program
  1 - Beta
  2 - Binomial
  3 - Negative Binomial
  4 - Chi Square
  5 - Non-central Chi-squared
  6 - F
  7 - Non-Central-F
  8 - Gamma
  9 - Normal
 10 - Poisson
 11 - T
 12 - Non-Central-T
 Enter Distribution:
 > 11

          T(TVAL,DF) = C,1-C          (T)
  Input TVAL, DF, and C or 1-C for t distribution.
    OR: type HELP then press ENTER key for help
    OR: Press ENTER key to change the distribution

> help

\end{verbatim}

\begin{commentary}

Typing   'help' within a distribution context causes a printout of
a brief description of the distribution and its parameters.  The 
description of the distribution is probably useful only to those
knowedgeable about statistics and even then only in cases (such as
the gamma or exponential) in which there are alternative
parameterizations
in the literature.

\end{commentary}

\begin{verbatim}


     The T Density is proportional to:
               (1+t**2/DF) ** (-(DF+1)/2)

  TVAL - Upper Limit of Integration of T Density
    DF - Degrees of Freedom
     C - Cumulative Distribution Function at X
   1-C - Complement of C

 Press the Return or Enter key to continue ...

>

          T(TVAL,DF) = C,1-C          (T)
  Input TVAL, DF, and C or 1-C for t distribution.
    OR: type HELP then press ENTER key for help
    OR: Press ENTER key to change the distribution

1.97 9 ? .

   ---- Parameters ---      ----- Answer ------      - P-val -
    TVAL        DF             C        1-C           2 Sided
    1.970000   9.000000      0.959829   0.040171      0.080342

          T(TVAL,DF) = C,1-C          (T)
  Input TVAL, DF, and C or 1-C for t distribution.
    OR: type HELP then press ENTER key for help
    OR: Press ENTER key to change the distribution

\end{verbatim}

\begin{commentary}

     The question mark in the third position indicates that C and 1-C
     are being calculated. The period in the fourth position indicates
     that C was entered and 1-C should be skipped. In this case
     1.97 9 . ? would have been equivalent.

\end{commentary}

\mysubsection{An Example of Tables of Values}

We continue  with the  t-distribution example showing  how a  table of
values can be created.

\begin{verbatim}

          T(TVAL,DF) = C,1-C          (T)
  Input TVAL, DF, and C or 1-C for t distribution.
    OR: type HELP then press ENTER key for help
    OR: Press ENTER key to change the distribution

> ? 9 t .

\end{verbatim}

\begin{commentary}

     The question mark as the first item indicates that TVAL is to be
     calculated. The equals sign indicates that the previous value of DF
     is to be used. The t indicates that a list of C values is to be 
     created and TVAL is to be calculated for each number on the list.
     The . indicates that the parameter C was entered and 1-C should be
     skipped.

\end{commentary}
                                   
\begin{verbatim}

Build List for C

 The list currently contains 0 values.
 The maximum number allowed is 100.

     Select an option:
        1 - ADD individually specified values to the list
        2 - ADD equally spaced values to the list
        3 - ADD logarithmically spaced values to the list
        4 - PRINT the list
        5 - DELETE a single element of the list
        6 - DELETE consecutive elements of the list
        7 - SORT the list in ascending order and eliminate duplicate values
        8 - QUIT modifying the list
 > 2

\end{verbatim}

\begin{commentary}

In order to have ``nice'' values, the number of intervals should
divide the high value minus the low evalue evenly.  In our example
8 divides (0.9-0.8).

\end{commentary}

\begin{verbatim}

 Enter beginning bound, ending bound and number of intervals.
 > 0.1 0.9 8
Build List for C

 The list currently contains 9 values.
 The maximum number allowed is 100.

     Select an option:
        1 - ADD individually specified values to the list
        2 - ADD equally spaced values to the list
        3 - ADD logarithmically spaced values to the list
        4 - PRINT the list
        5 - DELETE a single element of the list
        6 - DELETE consecutive elements of the list
        7 - SORT the list in ascending order and eliminate duplicate values
        8 - QUIT modifying the list
 > 8
   --------- Parameters ---------    - Answer -
      C         1-C        DF            TVAL
    0.100000   0.900000   9.000000     -1.383029
    0.200000   0.800000   9.000000     -0.883404
    0.300000   0.700000   9.000000     -0.543480
    0.400000   0.600000   9.000000     -0.260955
    0.500000   0.500000   9.000000     0.000E+00
    0.600000   0.400000   9.000000      0.260955
    0.700000   0.300000   9.000000      0.543480
    0.800000   0.200000   9.000000      0.883404
    0.900000   0.100000   9.000000      1.383029

\end{verbatim}

\mysection{Comments on Calulations for Discrete Distributions}

The binomial, negative binomial, and Poisson distributions are defined
on the integers.  That is:

\begin{itemize}

\item In the binomial distribution, the number of successes (S) and
the number of trials (N) take on integer values.

\item In the negative binomial distribution, the number of
  successes(S) and the number of failures (F) take on integer values.

\item In the Poisson distribution, the number of events takes on
integer values.

\end{itemize}

When any  of these discrete quantities  are calculated, it  is done by
embedding the distribution in  a continuous distribution: the binomial
and negative binomial are special  cases of the beta distribution; the
Poisson  distribution is  a special  case of  the gamma  distribution. 
These continuous distributions are  solved for the relevent quantities
which are converted  back to values for the  integer valued parameter. 
These values are ususally {\bf  not} integers but contain a fractional
part.   Since an  integer answer  is usually  desired,  STATTAB prints
three  lines of answers  for these  cases.  The  first line  shows the
non-integer  result; the second,  the result  for the  largest integer
smaller than the value; the third, the result for the smallest integer
larger  than  the   value.   Here  is  an  example   for  the  Poisson
distribution.

\begin{verbatim}

          P(N,MEAN) = C,1-C           (Poisson)
  Input N, MEAN, and C or 1-C for Poisson distribution.
    OR: type HELP then press ENTER key for help
    OR: Press ENTER key to change the distribution

? 10 t .

Build List for C

 The list currently contains 0 values.
 The maximum number allowed is 100.

     Select an option:
        1 - ADD individually specified values to the list
        2 - ADD equally spaced values to the list
        3 - ADD logarithmically spaced values to the list
        4 - PRINT the list
        5 - DELETE a single element of the list
        6 - DELETE consecutive elements of the list
        7 - SORT the list in ascending order and eliminate duplicate values
        8 - QUIT modifying the list
 > 2

 Enter beginning bound, ending bound and number of intervals.
 > 0.2 0.8 6
Build List for C

 The list currently contains 7 values.
 The maximum number allowed is 100.

     Select an option:
        1 - ADD individually specified values to the list
        2 - ADD equally spaced values to the list
        3 - ADD logarithmically spaced values to the list
        4 - PRINT the list
        5 - DELETE a single element of the list
        6 - DELETE consecutive elements of the list
        7 - SORT the list in ascending order and eliminate duplicate values
        8 - QUIT modifying the list
 > 8
   ---------- Parameters --------     - Answer -
      C         1-C        MEAN           N
    0.200000   0.800000  10.000000      6.798297
    0.130141   0.869859  10.000000      6.000000
    0.220221   0.779779  10.000000      7.000000
 ------------------------------------------------
    0.300000   0.700000  10.000000      7.724082
    0.220221   0.779779  10.000000      7.000000
    0.332820   0.667180  10.000000      8.000000
 ------------------------------------------------
    0.400000   0.600000  10.000000      8.543104
    0.332820   0.667180  10.000000      8.000000
    0.457930   0.542070  10.000000      9.000000
 ------------------------------------------------
    0.500000   0.500000  10.000000      9.331353
    0.457930   0.542070  10.000000      9.000000
    0.583040   0.416960  10.000000     10.000000
 ------------------------------------------------
    0.600000   0.400000  10.000000     10.141095
    0.583040   0.416960  10.000000     10.000000
    0.696776   0.303224  10.000000     11.000000
 ------------------------------------------------
    0.700000   0.300000  10.000000     11.030752
    0.696776   0.303224  10.000000     11.000000
    0.791556   0.208444  10.000000     12.000000
 ------------------------------------------------
    0.800000   0.200000  10.000000     12.101946
    0.791556   0.208444  10.000000     12.000000
    0.864464   0.135536  10.000000     13.000000
 ------------------------------------------------

\end{verbatim}

\mysection{Error Notification}

Stattab notifies the user of errors either in legal input values or if
the program cannot the requested  value.  A commonly occurring not too
obvious  case is  the  inverse of  a  discrete cdf  where the  desired
cumulative probabability  is too  small.  Because the  distribution is
discrete, the probability of 0 or fewer events (equals the probability
of 0  events) is positive, not  zero.  Hence, there is  no inverse cdf
for probabilities smaller than this value.  Here is an example of what
happens when such a value is requested.  The process of getting to the
binomial calculation is ommitted.

\begin{verbatim}

 Enter Distribution: 
 > 2
 
          Bin(S,N,P,1-P) = C,1-C      (Binomial)                               
  Input S, N, P or 1-P, and C or 1-C for binomial distribution.                
    OR: type HELP then press ENTER key for help
    OR: Press ENTER key to change the distribution
 
? 5 0.5 . 0.01 .
 
 --------------------- Parameters ---------------------    -- Answer --        
      C         1-C         N          P         1-P            S              
************************************************************************
    0.010000   0.990000   5.000000   0.500000   0.500000     0.000E+00          
Answer (if any) is BELOW the LOWER search bound
************************************************************************

\end{verbatim}

The probability of zero events  in five trials when the probability of
successs is 0.5 is 0.03125.  The number of successes needed to achieve
a smaller probability makes no sense.

The  program  returns 0  which  is the  lower  bound  of the  interval
searched for an  answer.  However, the error message  informs the user
that 0 is not a correct answer.

In some  (rare) cases an  answer may exist  but be above or  below the
endpoint  of the  interval searched.   The right  endpoint  is usually
$10^{10}$ if the distribution has an infinite range.

\mysection{Creating a Table of the Binomial Distribution}

We  illustrate the creation  of a  table of  the individual  terms and
cumulative distribution of the binomial distribution.

\begin{verbatim}

          Bin(S,N,P,1-P) = C,1-C      (Binomial)
  Input S, N, P or 1-P, and C or 1-C for binomial distribution.
    OR: type HELP then press ENTER key for help
    OR: Press ENTER key to change the distribution

t 10 0.5 . ? .

Build List for S

 The list currently contains 0 values.
 The maximum number allowed is 100.

     Select an option:
        1 - ADD individually specified values to the list
        2 - ADD equally spaced values to the list
        3 - ADD logarithmically spaced values to the list
        4 - PRINT the list
        5 - DELETE a single element of the list
        6 - DELETE consecutive elements of the list
        7 - SORT the list in ascending order and eliminate duplicate values
        8 - QUIT modifying the list
 > 2

 Enter beginning bound, ending bound and number of intervals.
 > 0 10 10
Build List for S

 The list currently contains 11 values.
 The maximum number allowed is 100.

     Select an option:
        1 - ADD individually specified values to the list
        2 - ADD equally spaced values to the list
        3 - ADD logarithmically spaced values to the list
        4 - PRINT the list
        5 - DELETE a single element of the list
        6 - DELETE consecutive elements of the list
        7 - SORT the list in ascending order and eliminate duplicate values
        8 - QUIT modifying the list
 > 8
  ------------- Parameters -------------   ------ Answer ------    -- Pdf --
     S         N         P        1-P          C        1-C          Term
  0.000E+00 10.000000  0.500000  0.500000    0.977E-03  0.999023    0.977E-03
   1.000000 10.000000  0.500000  0.500000     0.010742  0.989258     0.009766
   2.000000 10.000000  0.500000  0.500000     0.054687  0.945312     0.043945
   3.000000 10.000000  0.500000  0.500000     0.171875  0.828125     0.117188
   4.000000 10.000000  0.500000  0.500000     0.376953  0.623047     0.205078
   5.000000 10.000000  0.500000  0.500000     0.623047  0.376953     0.246094
   6.000000 10.000000  0.500000  0.500000     0.828125  0.171875     0.205078
   7.000000 10.000000  0.500000  0.500000     0.945312  0.054687     0.117188
   8.000000 10.000000  0.500000  0.500000     0.989258  0.010742     0.043945
   9.000000 10.000000  0.500000  0.500000     0.999023 0.977E-03     0.009766
  10.000000 10.000000  0.500000  0.500000     1.000000 0.000E+00    0.977E-03

\end{verbatim}

\mysection{ USING DSTATTAB TO OBTAIN P-VALUES}
     
\mysubsection{Introduction}
     
     Those using DSTATTAB to find p-values  will primarily be interested in
     the t (T),  normal  (N), chi-square (Chi),  and   variance  ratio  (F)
     distributions.  For  each of  these distributions, p-value information
     is printed when the user chooses the cumulative  distribution function
     (C) to be calculated.  A brief discussion  of  each of the use of each
     of  these distributions and  the printed p-value information  is given
     below.
     
\mysubsection{T Distribution}
     
     The t-test is used to test  (1)  whether the mean  of a sample from an
     assumed normal distribution is consistent with some specified value of
     the population mean, (2) whether there is evidence for a difference in
     the  population means  of  two  normally  distributed populations from
     which samples  have been obtained, or  (3)  whether there  is evidence
     that a  linear regression slope (again assuming  a normal distribution
     about the population line) differs from a predetermined value.
          
     Different experiments to the same  point can  illustrate these usages.
     Suppose that we are concerned with the effect of a  drug on the weight
     of animals.  The following three  experiments would shed some light on
     this issue.  (1) Take  a group of adult animals  whose weight will not
     vary greatly due  to  growth.  Weigh  them before they  are given  the
     drug, then put them on the drug for a month and weigh them again.  The
     difference after drug  weight  minus before drug  weight is the number
     recorded  for  each animal.   If   the drug  has no  effect  then  the
     population (infinite  sample size) mean is zero.   Testing whether the
     data is consistent with a mean of zero is the point of the experiment.
     (2) Two equal sized groups of animals  are used.  One group is  put on
     the drug for a month, the other is not.  The  difference in weights is
     again  recorded.  If  the  drug has no   effect,  the population  mean
     difference should be  zero.  (3) Several  groups of animals are placed
     on  different  doses of the  drug  for a month   and the difference in
     weights recorded.  This  difference is  regressed on  the dose to test
     for a dose effect.  If there is no effect, the population slope of the
     regression line is zero.
     

     In determining the p-value, the user has to  know whether  a one-sided
     or two-sided test is being conducted.  A two-sided test is appropriate
     when a change in either direction is  of interest  (i.e., it is  to be
     determined whether the drug causes weight loss, weight gain, or has no
     effect  on weight).  This  might be the case  were the drug  found  to
     contaminate animal feed.  A  one-sided test is appropriate when either
     a change in only one  direction is of interest or  when it  is known a
     priori that the  change can only be  in one  direction if there is any
     change  at all.  If a drug  were to  be marketed as a weight reduction
     potion for  humans and was undergoing tests   in animals, only  weight
     reduction would be  of interest.  No effect  or weight gain would both
     cause the drug not to be marketed.  Alternatively, the drug might be a
     slight modification of another drug known to  cause weight loss and it
     is known that it  can't cause weight  gain although it might   have no
     effect.
     
     Suppose experiment  (1) were performed  using 10 animals  and the mean
     before minus after value was 5.2  with an estimated  standard error of
     the  mean of 2.64; then  the  t value  is  5.2/2.64 = 1.97.   We  tell
     DSTATTAB that the TVAL is 1.97 and the degrees of freedom (DF) is 9 --
     one  less than  the  number  of  animals --  and  that  the cumulative
     distribution function (CDF) is  to be calculated. DSTATTAB returns the
     value  of the CDF   as 0.96  (or very  close to it)  and the Two-Sided
     P-Value as 0.08.  If  a  two-sided test is  being  used, 0.08  is  the
     p-value.
     
     If  a  one-sided test  is   being used and   the answer points  in the
     expected direction, then the p-value is half of the two-sided p-value,
     i.e., 0.04. If you don't want to bother with dividing a number by two,
     give DSTATTAB the value  of t  with a negative  sign, i.e.,  -1.97 and
     DSTATTAB returns C as the desired p-value, 0.04.
     
     If a one-sided  test  is being  used,  but the answer   points in  the
     opposite direction of that expected, then the p-value is one minus the
     value calculated in the previous paragraph,  or 0.96.  To avoid having
     to do these  calculations by hand,  enter  the  t value as a  positive
     number.
     
     The rule that  follows provides a  consistency   check on the  p-value
     obtained when a one-sided test is being used.  If the result points in
     the  expected direction, the  p-value is always  less than or equal to
     one-half; if the   result   points in the   unexpected direction,  the
     p-value is greater than or equal  to one-half.  If  your p-value is in
     the wrong half of the  0 to 1 interval,   try again with the  opposite
     sign on the t and take the value from C.

\mysubsection{Normal}
     
     
     The normal distribution is generally used for  p-values when assessing
     the  significance of  a  slope in   a  non-linear  regression  such as
     logistic     regression   or     Cox's  proportional  hazards   model.
     Asymptotically, the estimate of  the slope  divided by   its  standard
     error is normal.  When  only one independent  variable is used  in the
     regression, the above discussion for the t distribution  holds exactly
     for the normal  distribution.  If there  is more  than one independent
     variable, then individual coefficients  should always be assessed with
     a two-sided   test.   However, the likelihood   ratio test  is   to be
     preferred to   this normal  approximation,  although  several  popular
     computer programs  do  not  print enough information   to perform  the
     likelihood ratio test.
     
     The normal distribution is more often used to calculate the proportion
     of extreme values in a population than it is  to obtain p-values.  For
     example, suppose that a  large study  has shown  the average height of
     American men   to be  5'  10"  (=70")  with  a standard deviation (NOT
     standard  error  of  the mean)  of   7.3".  If  heights  are  normally
     distributed, what proportion of the population can be expected to have
     a height of  7'  1" (=85") or more.   Entering X=85, MEAN=70,  SD=7.3,
     DSTATTAB yields  0.98  as  the value  of  the cumulative  distribution
     function.  But  the cdf is  the  proportion less than the  specified X
     value, so 0.02 of the population is taller than 7' 1".
     
     
\mysubsection{ Chi-Square}
     
     The chi-square  distribution is popularly  used for two purposes.  (1)
     In a  table of counts  of individuals cross-tabulated according to two
     characteristics   to  assess the    acceptability  of  the  equivalent
     hypotheses:  (a) the  proportion of each  entry in  the  table  is the
     product of  the two marginal  frequencies  of that entry;  (b) the row
     proportions are the same; and (c) the column proportions are the same.
     Usually, only  one of these three equivalent   formulations  will make
     sense in a  given experimental situation.  (2)  The second popular use
     for   the  chi-square distribution   is  to assess   agreement between
     categorized data and frequencies  predicted  by  some model.   STATTAB
     prints p-values whenever   the  cumulative distribution  function   is
     calculated for  the chi-square  statistic.   The chi-square  statistic
     assesses  deviations in  any  direction from  equality,  and there are
     generally many such  directions.    The  p-value provided  by  STATTAB
     applies to  these many directions.   Only in the  two by  two table in
     which one proportion is being tested against another is it possible to
     speak of a one-sided test.
     
     The chi-square distribution has  the   property  that  the larger  the
     value, the smaller the  p-value.  The p-value  is,  in fact, one minus
     the cumulative distribution function.
     
     If  the proportions  are in the  expected  direction, then a one-sided
     test would yield a p-value that  is half that listed by DSTATTAB.   If
     the proportions  are in the  unexpected direction, then the p-value is
     one minus half the p-value printed  by DSTATTAB.  The rule given above
     still holds:  if  the evidence points in  the  expected direction, the
     p-value is less than or  equal to 0.5; if  the evidence points  in the
     other direction the p-value is at least 0.5.
     
\mysubsection{ Variance Ratio (F)}
     
     The F  distribution is used in the  analysis of variance  to  test the
     hypothesis of  equality  of means of  several groups drawn from normal
     distributions  with the same  standard deviation.  It is also  used to
     assess  the  significance  of one   or more  variables in   a   linear
     regression (again assuming a normal  distribution of values about  the
     line).  When   there   are more than two  groups   or more   than  one
     independent variable being assessed in a regression, the assessment is
     inherently  multi-directional  and    no simple  calculation    yields
     one-sided p-values.
     
     As is the case  with  the chi-square  distribution, the p-value is one
     minus the cumulative distribution function.
   
     If the slope of only one independent  variable is  being assessed in a
     regression or if  only   two  groups are  compared in   an analysis of
     variance then a one-sided test makes sense.  (In these cases, the test
     is precisely the same as that provided by the t.)   If the results are
     in the  expected direction  and a one-sided  test is   used,  then the
     appropriate p-value is  half that printed by  DSTATTAB. If the results
     are in  the opposite of  the  expected direction, the  p-value  is one
     minus one-half of the value printed by DSTATTAB.  The same  rule about
     p-values being greater or less than one-half applies here.

\mysection{A BRIEF GUIDE TO THE AVAILABLE DISTRIBUTIONS}
     
         The  parameters and  densities of  the  various distributions  are
     listed  here  for reference by  the knowledgeable.  This is  necessary
     because there are several different ways of  parameterizing several of
     these distributions, particularly the beta and gamma.  The information
     contained in  this section  is  also  available  in the   Stattab help
     facility.
     
\mysubsection{Beta Distribution}

{\bf \centering Beta(X,1-X,A,B)=C,1-C.\\}

\begin{center}
\begin{tabular}{rl}
           X & Upper Limit of Integration of Beta Density\\
         1-X & Complement of X\\
           A & First Parameter of Beta Distribution\\
           B & Second Parameter of Beta Distribution\\
           C & Cumulative Distribution Function at X\\
         1-C & Complement of C\\
\end{tabular}
\end{center}
     The beta density in t is proportional to
\[ t^{A-1} (1-t)^{B-1} \]

\mysubsection{Binomial Distribution}
     
{\bf \centering Bin(S,N,P,1-P)=C,1-C.\\}
\begin{center}
\begin{tabular}{rl}
           S & Number of Successes\\
           N & Number of Trials\\
           P & Probability of Success Each Trial\\
         1-P & Probability of Failure Each Trial (1-P)\\
           C & Probability of 0 to S Successes\\
         1-C & Complement of C\\
\end{tabular}
\end{center}

     The binomial density in S is proportional to 
\[ P^S (1-P)^{N-S} \].


\mysubsection{Negative Binomial Distribution}
     
{\bf \centering Negbin(X,S,P,1-P)=C,1-C.\\}

\begin{center}
\begin{tabular}{rl}
            X & Number of Failures\\
            S & Number of Successes\\
            P & Probability of Success Each Trial\\
          1-P & Probability of Failure Each Trial (1-P)\\
            C & Probability of 0 to M Failures Before S Successes\\
          1-C & Complement of C\\
\end{tabular}
\end{center}

     The individual term of the negative binomial is the probability of
     X failures before S successes and is proportional to
          \[ P^S  (1-P)^X \]
  
   
\mysubsection{Chi-Square Distribution}
     
{\bf \centering Chi(X,DF)=C,1-C.\\}

\begin{center}
\begin{tabular}{rl}
          X  & Upper Limit of Integration of Chi-Square Density\\
          DF & Degrees of Freedom\\
          C  & Cumulative Distribution Function at X\\
         1-C & Complement of C\\
\end{tabular}
\end{center}
     
     The Chi-Square Density in t is proportional to 
\[t^{df/2 - 1}  \exp(-t/2).\]
     
  
\mysubsection{Noncentral Chi-square Distribution}
     
{\bf \centering Chn(X,DF,PNONC)=C,1-C.\\}

\begin{center}
\begin{tabular}{rl}
       X     & Upper Limit of Integration of Non-Central Chi-Square Density\\
       DF    & Degrees of Freedom\\
       PNONC & Noncentrality Parameter\\
       C     & Cumulative Distribution Function at X\\
       1-C   & Complement of C\\
\end{tabular}
\end{center}

The noncentral chi-square distribution can be explained as follows.
Let $d_1, d_2, \ldots, d_{DF}$ be numbers such that
\[ PNONC = \sum_{i=1}^{DF} d_{i}^{2}. \]
Let $N_1, N_2, \ldots, N_{DF}$ be independent unit normal deviates.
Then the noncentral chi-square distribution is the distribution of
\[ \sum_{i=1}^{DF} ( N_i + d_i ).\]
The density can only be written as an infinite series.
     
\mysubsection{Variance Ratio (F) Distribution}
     
{\bf \centering F(X,DFN,DFD)=C,1-C.\\}

\begin{center}
\begin{tabular}{rl}
         X   & Upper Limit of Integration of the F Density\\
         DFN & Degrees of Freedom of the Numerator\\
         DFD & Degrees of Freedom of the Denominator\\
         C   & Cumulative Distribution Function at X\\
         1-C & Complement of C\\
\end{tabular}
\end{center}
     
     The F Density in t is proportional to
\[ t^{(DFN-2)/2} (DFD+DFN*t)^{-(DFN+DFD)/2}. \]
     
     
\mysubsection{Non-Central Variance Ratio (F) Distribution}


{\bf \centering F(X,DFN,DFD,PNONC)=C,1-C.\\}

\begin{center}
\begin{tabular}{rl}
          X     & Upper Limit of Integration of the Non-central F Density\\
          DFN   & Degrees of Freedom of the Numerator\\
          DFD   & Degrees of Freedom of the Denominator\\
          PNONC & Noncentrality Parameter\\
          C     & Cumulative Distribution Function at X\\
          1-C   & Complement of C\\
\end{tabular}
\end{center}

The   non-central  F   distribution   is   the   distribution of    \[
\frac{\chi^{'2}_{DFN}(PNONC)}{DFN} (\frac{\chi^{2}_{DFD}}{DFD})^{-1} \] where
$\chi^{'2}_{DFN}(PNONC)$   is distributed as  a  non-central chisquare
with degrees of freedom, DFN,  and noncentrality parameter PNONC and $
\chi^{2}_{DFD} $ is distributed as a  central chisquare with DFD degrees
of freedom.  The density is usually expressed as an infinite series.

\mysubsection{Gamma Distribution}
     
{\bf \centering Gamma(X,A,B)=C,1-C.\\}

\begin{center}
\begin{tabular}{rl}
         X  & Upper Limit of Integration of the Gamma Density\\
         A  & Scale\\
         B  & Shape\\
         C  & Cumulative Distribution Function at X\\
       1-C  & Complement of C\\
\end{tabular}
\end{center}
     
     The Gamma Density in t is proportional to
\[ t^{R-1}  \exp(-A*t) \] 
     
     
\mysubsection{Normal Distribution}
     
{\bf \centering N(X,MEAN,SD)=C,1-C\\}

\begin{center}
\begin{tabular}{rl}
        X    & Upper Limit of Integration of Normal Density\\
        MEAN & Mean of Normal Distribution\\
        SD   & Standard Deviation of Normal Distribution\\
        C    & Cumulative Distribution Function at X\\
        1-C  & Complement of C\\
\end{tabular}
\end{center}
     
     The Normal Density is Proportional to
\[     /exp( - 0.5  (\frac{X-MEAN}{SD})^2). \] 

     
     
\mysubsection{Poisson Distribution}

{\bf \centering P(N,MEAN)=C,1-C\\}

\begin{center}
\begin{tabular}{rl}
        N    & Number of Events\\
        MEAN & Average Number of Events\\
        C    & Probability of 0 to N Events\\
        1-C  & Complement of C\\
\end{tabular}
\end{center}

     The Poisson density is proportional to 
\[ \frac{MEAN^N}{N!}  \]
     
     
\mysubsection{Student's t Distribution}

{\bf \centering T(TVAL,DF)=C,1-C\\}

\begin{center}
\begin{tabular}{rl}
        TVAL & Upper Limit of Integration of T Density\\
        DF   & Degrees of Freedom\\
        C    & Cumulative Distribution Function at X\\
        1-C  & Complement of C\\
\end{tabular}
\end{center}

     The T Density is proportional to
\[     (1+\frac{t^2}{DF})^{(-(DF+1)/2)} \]

\mysubsection{Non-central T}

{\bf \centering TNC(TVAL,DF,PNONC)=C,1-C}

\begin{center}
\begin{tabular}{rl}
        TVAL & Upper Limit of Integration of T Density\\
        DF   & Degrees of Freedom\\
        PNONC&Noncentrality parameter\\
        C    & Cumulative Distribution Function at X\\
        1-C  & Complement of C\\
\end{tabular}
\end{center}

The noncentral T distribution is the distribution of
\[ \frac{N+PNONC}{\chi_{DF} DF^{-1/2}} \]
where $\chi_{DF}$ is distributed as the square root of 
a central chi-square with DF degrees of freedom.

\end{document}


